\documentclass{resume}
\usepackage{zh_CN-Adobefonts_external} 
\usepackage{linespacing_fix}
\usepackage{cite}
\begin{document}
\pagenumbering{gobble}



%***"%"后面的所有内容是注释而非代码,不会输出到最后的PDF中
%***使用本模板,只需要参照输出的PDF,在本文档的相应位置做简单替换即可
%***修改之后,输出更新后的PDF,只需要点击Overleaf中的“Recompile”按钮即可
%**********************************姓名********************************************
\name{干皓軍}
%**********************************联系信息****************************************
%第一个括号里写手机号,第二个写邮箱
\contactInfo{(+886) 985463215}{kanhaojun.ascdfkan@gmail.com}
%**********************************其他信息****************************************
%在大括号内填写其他信息,最多填写4个,但是如果选择不填信息,
%那么大括号必须空着不写,而不能删除大括号。
%\otherInfo后面的四个大括号里的所有信息都会在一行输出
%如果想要写两行,那就用两次这个指令(\otherInfo{}{}{}{})即可
%\otherInfo{性别:男}{籍贯:江南}{}{}
%\otherInfo{来历:钱钟书《围城》}{}{}{}
%*********************************照片**********************************************
%照片需要放到images文件夹下,名字必须是you.jpg,如果不需要照片可以不添加此行命令
%0.15的意思是,照片的宽度是页面宽度的0.15倍,调整大小,避免遮挡文字
\yourphoto{0.125}
%**********************************正文**********************************************


%***大标题,下面有横线做分割
%***一般的标题有:教育背景,实习(项目)经历,工作经历,自我评价,求职意向,等等


%***********一行子标题**************
%***第一个大括号里的内容向左对齐,第二个大括号里的内容向右对齐
%***\textbf{}括号里的字是粗体,\textit{}括号里的字是斜体

\datedsubsection{\textbf{銘傳大學},資訊管理學系,\textit{學士}}{2015.08 - 20.06}
\datedsubsection{\textbf{銘傳大學},資訊工程學系,\textit{碩士}}{2015.08 - 2017.06}
\datedsubsection{\textbf{龍華科技大學},資訊管理學系,\textit{學士}}{2011.08 - 2015.06}
% \begin{itemize} [parsep=1ex]
%   \item \textbf{書卷獎}:(2012.11)
%   \item \textbf{學程}:資訊金融學程
% \end{itemize}
%***********列举*********************
%***可添加多个\item,得到多个列举项,类似的也可以用\textbf{}、\textit{}做强调
%\begin{itemize} [parsep=1ex]
%  \item \textbf{证书来源}:购买自爱尔兰商人
%\end{itemize}

%\datedsubsection{\textbf{銘傳大學龍華科技大學},,\textit{碩士}}{2021.09 - 至今}
% \begin{itemize} [parsep=1ex]
%   \item \textbf{書卷獎}:(2024.04)
% \end{itemize}
\section{社團與校園經歷}
\datedsubsection{\textbf{Summer School Ming Chuan University- Michigan Location
.	暑期學術計劃 - 銘傳大學密西根校區兼美國塞基諾州立大學}}{ 2018.7 - 2018.8}
\begin{itemize}[parsep=0.5ex]
  \item 參加銘傳大學美國分校的學術交流活動與瞭解美國社會的歷史與文化還有社會現狀。
\end{itemize}

\datedsubsection{\textbf{SITCON Summer Camp , SITCON 學生電腦年會兼夏令營 - 學生電腦組織與開放原始碼活動}}{2017.1 - 2017.8}
\begin{itemize}[parsep=0.5ex]
  \item 參加 SITCON 的夏令營活動,學習基本的 Python 技能與瞭解更進階的電腦原理。

  \item SITCON 是臺灣最大的學生開放原始碼的電腦組織,由臺灣多所大學的電腦學生社團所組成。由此接觸開放原始碼的技術社群,由此獲得多項電腦技能。
\end{itemize}

\datedsubsection{\textbf{LHUIOSC 龍華科技大學的開放原始碼的學生社群, }}{2015.1 – 2015. 8}
\begin{itemize}[parsep=0.5ex]
  \item	參加龍華科技大學的開放原始碼社團,進而參與 COSCUP 活動。COSCUP 活動則是臺灣最大的開放原始碼社群活動。同時在此瞭解類 UNIX OS 的指令操作。
\end{itemize}

\section{工作經歷}

\datedsubsection{\textbf{Compulsory Military Service},二兵}{2021.9 - 2022.2}
% \begin{itemize}[parsep=0.5ex]
%   \item 維護
% \end{itemize}

% \datedsubsection{\textbf{JTaiwan},工程師}{2014.05 - 2014.08}
% \begin{itemize}[parsep=0.5ex]
%   \item 維護
% \end{itemize}


\section{專業技能}

\begin{itemize}[parsep=0.5ex]
  \item 程式語言 :  R, Java, JavaScript, Python, C/C++, Visual Basic, HTML/CSS, SQL(MSSQL, MariaDB), LaTeX
  \item 機器學習網站 : TensorFlow
  \item 作業系統 : Ubuntu, RedHat, Windows Server
  \item 專案管理 : Git, GitHub
\end{itemize}

\section{專業證照}

\begin{itemize}[parsep=0.5ex]
  \item Techficiency Quotient Certification Excel 17, 2020.1
  \item Enterprise Resouce Planning System, 2019.1
%  \item Novell Certified Linux Administrator (NCLA 11), 2014
%  \item Oracle Certified Expert (OCE 10g), 2014
%  \item Oracle Certification for Java Programmer (OCJP 6), 2013
%  \item IPMA Certified Project Management Associate (Level D), 2013
\end{itemize}

\section{研究與專案}

\datedsubsection{\textbf{Whack-a-mole Game}}{2020.04 - 2020.07}
\begin{itemize}[parsep=0.5ex]
  \item 描述: 這是一款使用 Pixel 3 API 30 來開發的遊戲,其功能包括倒數計時和賺取積分,遊戲結束後,玩家可以看到自己的排名和最終得分。
  \item 程式語言和軟體使用 : Android、Java
\end{itemize}
% Autopay

\datedsubsection{\textbf{Autopay}}{2019.04 - 2019.06}
\begin{itemize}[parsep=0.5ex]
  \item 描述: 這是一個用 Android Studio IDE 建構的系統。它允許使用者設定未來付款的日期。使用者只需輸入付款金額並確認,即可輸入有關付款的資訊將出現在畫面上。付款處理完成後,使用者可以手動刪除該條目,下次再重複此程序。

  \item 程式語言和軟體使用 : Android、Java、Kotlin
\end{itemize}

\datedsubsection{\textbf{Ideallife}}{2013.08 - 2015.08}
\begin{itemize}[parsep=0.5ex]
  \item 描述 : 建立了一個資料庫來確定哪些城市適合人們居住,並設計了使用者介面,透過 Web Server 對溫度、濕度、空氣品質、交通方式等因素進行統計,收集來進行分析。使用者只需點擊想要的位置,所有相關資訊就會出現顯示在螢幕上。同時基於資料視覺化的概念,使用R和Shiny Server開發。目標是分析適合老年人口的城市和視覺化平台。
  \item 程式語言和軟體使用 : R、R Shiny Server、Ubuntu Server、資料視覺化
\end{itemize}
\end{document}
